% Metavars
\newcommand\term{t}
\newcommand\ty{\tau}
\newcommand\tyenv{\Gamma}
\newcommand\csenv{C}

% Term
\newcommand\var{x}
\newcommand\app[2]{#1\,#2}
\newcommand\lam[2]{\lambda #1. #2}

% Value
\newcommand\val{v}

% Type
\newcommand\tyconst{\iota}
\newcommand\tyvar{\alpha}
\newcommand\tykind{*}
\newcommand\tyarrow[2]{#1 \rightarrow #2}
% Datatype
\newcommand\dtype{\epsilon \, \vec{\ty}}

% Binding
\newcommand\binding[2]{(#1 : #2)}

% Typing environment
\newcommand\tyenvnil{\bullet}
\newcommand\tyenvcons[3]{#1; \binding{#2}{#3}}

% Jugements
\newcommand\tycheck[3]{#1 \vdash #2 : #3}
\newcommand\tyenvmem[2]{#1 \in #2}

% Keyword
\newcommand\kw[1]{\operatorname{#1}}

% Pattern
\newcommand\pator{\;|\;}

% Operation semantic
\newcommand\redto[2]{#1 \longrightarrow #2}

% Naming rules
\newcommand\rname[1]{\;\;\textsc{{\small(#1)}}}
\newcommand\rref[1]{\textsc{\small#1}}
\newcommand\rcase[1]{\paragraph{Cas \rref{#1}:}}

\newcommand\freeinenv[1]{#1 \# \tyenv}
\newtheorem{thm}{Théorème}
\newtheorem{lemma}{Lemme}

\section{Formalisation du langage}

Le langage développé reprend le cœur du lambda-calcul polymorphe
\cite{Reynolds94anintroduction} (aussi connu sous le nom de \emph{Système F})
et l'étend avec les notions de primitives et de \emph{bloc} muni d'un
\emph{tag} et les constructions d'analyse de tag et d'aiguillage.

\subsection{Syntaxe}

Présentons tout d'abord la syntaxe de ce langage.

\subsubsection{Lambda-calcul polymorphe} 

Les trois premières constructions du langage sont les variables, l'application
et l'abstraction. Elles forment le lambda-calcul simplement typé.

$$\synp{\term_1, \term_2} \var, y$$
%
Référence à une variable (liée dans l'environnement).

$$\synp{\term_1, \term_2}   \term_1 \, \term_2$$
%
Application d'un terme à un autre.

$$\synp{\term_1, \term_2}   \lambda \binding{\var}{\ty} . \term$$
%
Abstraction par rapport à une variable; $\var$ est liée dans $\term$.

Ces trois constructions permettent de construire et d'appliquer des fonctions,
dont le type est noté par une \emph{flèche} paramétrée par le domaine et le
codomaine de la fonction.

L'application et l'abstraction de types rajoutent le polymorphisme, représenté
au niveau des types par la quantification universelle.

$$\synp{\term_1, \term_2}   \lambda \tyvar . \term$$
%
Cette construction lie $\tyvar$ au sein du terme $\term$.

$$\synp{\term_1, \term_2}   \term \, \ty $$
%
Contrairement à un langage tel qu'OCaml, la généralisation et l'instantiation
des valeurs polymorphes sont marquées explicitement. Si un tel langage devait
être utilisé dans le compilateur, c'est lors de la traduction que celles-ci
seraient ajoutées par le \emph{front-end}.

\subsubsection{Primitives}

Des primitives peuvent être appliquées via la construction :
%
$$\synp{\term_1, \term_2} prim \, \vec{\term}$$

Leur comportement est proche de celui de fonctions, mais elles ne sont pas
définissables dans le langage. Leurs règles de réduction et de typage seront
explicitées après.

Les primitives sont toujours invoquées avec tous leurs arguments (il n'y a pas
d'applications partielles).

\begin{description}
  \item[$\kw{makeblock}_{n,\dtype}$] Allocation d'un bloc avec l'étiquette $n$
    et le type $\dtype$.
  \item[$\kw{field}_n$] Projection du n-ième champ d'un bloc.
  \item[$\kw{eq}_z$] Test d'égalité avec la constante $z$.
  \item[$\kw{lt}_z$] Test d'infériorité par rapport à la constante $z$.
  \item[$\kw{isint}$] Test si une valeur est une constante ou un bloc.
  \item[$\kw{isout}_{[z1,z2]}$] Test si une valeur entière est dans un
    intervalle.
\end{description}

\subsubsection{Extraction d'étiquette} 

$$\synp{\term_1, \term_2} \kw{unpack} \term_1 \kw{as} y : \var \kw{in} \term_2$$
%
$\var$ et $y$ sont liés dans $\term_2$ :
\begin{itemize}
  \item $y$ est lié à la valeur de l'étiquette de $\term_1$.
  \item $\var$ devient un alias de $\term_1$, mais dont le type est raffiné en
    fonction des tests effectués sur $y$.
\end{itemize}

\subsubsection{Aiguillage} 

$$\synp{\term_1, \term_2} \kw{switch} \term \kw{in} \vec{b} : \ty$$
%
L'étiquette de $\term$ est testée contre les branches de $\vec{b}$.
L'exécution se poursuit sur la branche qui satisfait le test.

Les branches ont les formes suivantes :
%
$$\synp{b} n \rightarrow \term \; | \; b$$
%
Le test réussit si l'étiquette est égale à $n$.
%
$$\synp{b}  \_ \rightarrow \term$$
%
Cas par défaut, le test réussit toujours.
%
$$\synp{b} \emptyset$$
%
Tous les cas ont été couverts. Si l'exécution devait arriver sur ce cas, ce
serait une erreur. En particulier, le système de type doit garantir que ce cas
ne peut arriver à l'exécution.

\subsubsection{Constructeurs de types}

% TODO Phi, ar, dom
\begin{align*}
  \syn{\rho} k \rightarrow C, \vec{\ty} \,;\, \rho
  \synor \ldots
  \synor \emptyset
\end{align*}

Soit $\epsilon$ l'ensemble des noms des constructeurs de types. \\
On suppose l'existence d'une fonction 
$\delta : \epsilon \times \vec{\ty} \to \rho$.

\todo{Meta constructeurs : IsInt, IsOut, Lt, Eq}

\subsubsection{Algèbre de type}

$$\synp{\ty_1, \ty_2} \tyvar, \beta$$
%
Référence à des variables de types.

$$\synp{\ty_1, \ty_2}  \ty_1 \rightarrow \ty_2$$
%
Type d'une fonction de $\ty_1$ vers $\ty_2$.

$$\synp{\ty_1, \ty_2}  \forall \tyvar. \ty$$
%
Type polymorphe, abstrait par rapport à $\tyvar$ qui est lié dans $\ty$.

$$\synp{\ty_1, \ty_2}  \dtype$$
%
$\epsilon$ est un nom de constructeur de type, dont les paramètres sont
instanciés avec les types de $\vec{\ty}$.

$$\synp{\ty_1, \ty_2}  \{ \var \}_{\dtype}$$
%
Type singleton : c'est le type exact de la variable nommée $\var$. Celui-ci est
affiné par les contraintes introduites dans l'environnement.

$$\synp{\ty_1, \ty_2}  \{ n: \vec{\ty} \}$$
%
Type du bloc d'étiquette $n$ et dont les valeurs ont les types de $\vec{ty}$.

\subsubsection{Définition des contraintes}

$$\synp{C} C \wedge C$$
%
Les deux contraintes doivent être vérifiées simultanément.

$$\synp{C}  \ty_1 = \ty_2$$
%
Cette contrainte impose que deux types soient égaux.

$$\synp{C}  \kw{tag} \var \in S$$
%
Cette contrainte impose que l'étiquette de $\var$ appartienne à l'ensemble $S$.
$\var$ doit être de type singleton. $S$ est un ensemble fini ou cofini
d'entiers.

$$\synp{C}  \kw{tag} \var = y$$
%
La valeur de la variable $y$ est égale à l'étiquette de la valeur de la
variable $\var$.

$$\synp{C}  \top$$
%
Contrainte toujours vérifiée.

$$\synp{C}  \bot$$
%
Contrainte jamais vérifiée.

\begin{figure}
\begin{align*}
  \syn{\term_1, \term_2} \var, y 
    \syndesc{variable}
  \synor      \term_1 \, \term_2
    \syndesc{application}
  \synor      \lambda \binding{\var}{\ty} . \term
    \syndesc{abstraction}
  \synor      \lambda \tyvar . \term
    \syndesc{abstraction de type}
  \synor      \term \, \ty
    \syndesc{application de type}
  \synor      prim \, \vec{\term}
    \syndesc{application de primitives}
  \synor      \kw{switch} \term \kw{in} \vec{b} : \ty
    \syndesc{aiguillage}
  \synor      \kw{unpack} \term_1 \kw{as} y : \var \kw{in} \term_2
    \syndesc{ouverture d'un bloc}
\end{align*}
\caption{Syntaxe des termes}
\end{figure}

\begin{figure}
\begin{align*}
  \syn{v}     \lambda \binding{\var}{\ty} . \term
    \syndesc{abstraction}
  \synor      \Lambda \tyvar . v
    \syndesc{abstraction de type}
  \synor      \{ n: \vec{v} \}
    \syndesc{bloc immédiat}
\end{align*}
\caption{Syntaxe des valeurs}
\end{figure}

\begin{figure}
\begin{align*}
  \syn{b} n \rightarrow \term \; | \; b
    \syndesc{cas constant}
  \synor  \_ \rightarrow \term
    \syndesc{cas par défaut}
  \synor  \emptyset
    \syndesc{absence de cas par défaut}
\end{align*}
\caption{Clauses de branchement}
\end{figure}

\begin{figure}
\begin{align*}
  \syn{\ty_1, \ty_2} \tyvar, \beta
    \syndesc{variables de type}
  \synor  \ty_1 \rightarrow \ty_2
    \syndesc{type flèche}
  \synor  \forall \tyvar. \ty
    \syndesc{quantification universelle}
  \synor  \dtype
    \syndesc{constructeur paramétré}
    \synor  \{ \var \}_{\dtype}
    \syndesc{type \emph{singleton}}
  \synor  \{ n: \vec{\ty} \}
    \syndesc{type d'un bloc immédiat}
\end{align*}
\caption{Syntaxe des types}
\end{figure}

\begin{figure}
\begin{align*}
  \syn{C} C \wedge C
    \syndesc{conjonction}
  \synor  \ty_1 = \ty_2
    \syndesc{égalité de types}
  \synor  \kw{tag} \var \in S
    \syndesc{restriction d'un \emph{tag}}
  \synor  \kw{tag} \var = y
    \syndesc{étiquette décrite par une variable}
  \synor  \top
    \syndesc{Vrai}
  \synor  \bot
    \syndesc{Faux}
\end{align*}
\caption{Syntaxe des contraintes}
\end{figure}

\begin{figure}
\begin{align*}
  \syn{S}
    \syndesc{ensemble fini}
  \synor  \lnot \{ k_1, \ldots, k_n \}
    \syndesc{ensemble cofini}
\end{align*}
\caption{Ensemble d'étiquettes}
\end{figure}

\begin{figure}
\begin{align*}
  \syn{\tyenv} .
    \syndesc{contexte vide}
  \synor \tyenvcons\tyenv{\var}{\ty}
    \syndesc{liaison de terme}
  \synor \tyenvcons\tyenv{\alpha}{*}
    \syndesc{liaison de type}
\end{align*}
\caption{Syntaxe des contextes}
\end{figure}

\begin{figure}
\begin{align*}
  \syn{prim} \kw{makeblock}_{n,\dtype}
    \syndesc{allocation d'un bloc}
  \synor \kw{field}_{n}
    \syndesc{projection d'un champ}
  \synor \kw{eq}_{z}
    \syndesc{égalité}
  \synor \kw{lt}_{z}
    \syndesc{infériorité}
  \synor \kw{isint}
    \syndesc{test d'entier}
  \synor \kw{isout}_{[z1,z2]}
    \syndesc{test d'intervalle}
  %\synor \kw{shift}_{z}
  %  \syndesc{décalage constant}
\end{align*}
\caption{Primitives du langage}
\end{figure}

\pagebreak

\subsection{Sémantique opérationnelle}

\begin{mathpar}
%
\infer{}{
  \redto{\app{(\lambda \binding{\var}{\ty} . \term)}{\val}}
        {\term\{ \var \leftarrow \val\}}
}\rname{R-App}
\\

\infer{
  \redto{\term_1}{\term'_1}
}{
  \redto{\app{\val}{\term_1}}{\app{\val}{\term'_1}}
}\rname{R-App-1}

\infer{
  \redto{\term_1}{\term'_1}
}{
  \redto{\app{\term_1}{\term_2}}{\app{\term'_1}{\term_2}}
}\rname{R-App-2}
\\

\infer{
  \redto{\term}{\term'}
}{
  \redto{\term \, \ty}{\term' \, \ty}
}\rname{R-TApp-1}

\infer{}{
\redto{(\Lambda \tyvar . \term) \ty}{\term \{ \tyvar \leftarrow \ty \}}
}\rname{R-TApp}
%
\end{mathpar}
%
\begin{mathpar}
%
\infer{
  \redto{\term_1}{\term'_1}
}{
  \redto{\kw{switch} \term_1 \kw{in} \vec{b}}{\kw{switch} \term'_1 \kw{in} \vec{b}}
}\rname{R-Switch-1}

\infer{}{
  \redto{\kw{switch} \val \kw{in} \_ \rightarrow \term}
        {\term}
}\rname{R-PatAny}
\\
%
\infer{
  \val = \{k : \vec{v}'\}
}{
  \redto{\kw{switch} \val \kw{in} k \rightarrow \term \pator \vec{b}}
        {\term}
}\rname{R-PatMatch}

\infer{
  \val = \{k' : \vec{v}'\} \\
  k \neq k'
}{
  \redto{\kw{switch} \val \kw{in} k \rightarrow \term \pator \vec{b}}
        {\kw{switch} \val \kw{in} \vec{b}}
}\rname{R-PatFail}
%
\end{mathpar}
%
\begin{mathpar}
%
\infer{
  \redto{\term_1}{\term'_1}
}{
  \redto{\kw{unpack} \term_1 \kw{as} \var \kw{in} \term_2}
        {\kw{unpack} \term'_1 \kw{as} \var \kw{in} \term_2}
}\rname{R-Unpack-1}
\\

\infer{
  \redto{\term_2}{\term'_2}
}{
  \redto{\kw{unpack} \val \kw{as} \var \kw{in} \term_2}
        {\kw{unpack} \val \kw{as} \var \kw{in} \term'_2}
}\rname{R-Unpack-2}

\infer{}{
  \redto{\kw{unpack} \val_1 \kw{as} \var \kw{in} \val_2}
        {\val_2\{\var \leftarrow \val\}}
}\rname{R-Unpack}
\\

\infer{
  \redto{\vec{\term}}{\vec{\term'}}
}{
  \redto{\app{prim}{\vec{\term}}}{\app{prim}{\vec{\term'}}}
}\rname{R-Prim}
\\

\end{mathpar}

\subsubsection{Réduction des primitives}
%
\begin{mathpar}
%
\infer{}{
  \redto{\kw{makeblock}_{k,\ty} \vec{v}}{\{k: \vec{v}\}}
}\rname{R-makeblock}

\infer{}{
  \redto{\kw{field}_i \{k: \vec{v}\}}{v_i}
}\rname{R-field}
%
%\infer{}{
%  \redto{\kw{shift}_z k}{z + k}
%}\rname{R-shift}

\infer{ k = z }
      { \redto{\kw{eq}_z k}{1} }
\rname{R-eq-1}

\infer{ k \neq z }
      { \redto{\kw{eq}_z k}{0} }
\rname{R-eq-0}

\infer{ k \le z }
      { \redto{\kw{lt}_z k}{1} }
\rname{R-less-1}

\infer{ k \nless z }
      { \redto{\kw{lt}_z k}{0} }
\rname{R-less-0}

\infer{}{
  \redto{\kw{isint} k}{1}
}\rname{R-isint-1}

\infer{}{
  \redto{\kw{isint} \{k: \vec{v}\}}{0}
}\rname{R-isint-0}

\infer{ k \notin [k_1,k_2] }
      { \redto{\kw{isout}_{[k_1,k_2]} k}{1} }
\rname{R-isout-1}

\infer{ k \in [k_1,k_2] }
      { \redto{\kw{isout}_{[k_1,k_2]} k}{0} }
\rname{R-isout-0}

\end{mathpar}

\subsection{Règles de bonnes formation}

\subsubsection{Formation des contextes}
\begin{mathpar}
\infer{ }{ \vdash . }

\infer{
  \freeinenv{\var} \\
  \vdash \tyenv \\
  \tyenv \vdash \ty
}{
  \vdash \tyenvcons\tyenv{\var}{\ty}
}

\infer{
  \freeinenv{\var} \\
  \vdash \tyenv \\
  \tyenv \vdash {\dtype}
}{
  \vdash \tyenvcons\tyenv{\var}{\{ \var \}_{\dtype}}
}
\end{mathpar}

\subsubsection{Formation des types}
\begin{mathpar}
%
\infer{
  \vdash \tyenv \\
  \binding{\tyvar}{*} \in \tyenv
}{
  \tyenv \vdash \tyvar
}

\infer{
  \tyenv \vdash \ty_1 \\
  \tyenv \vdash \ty_2
}{
  \tyenv \vdash \ty_1 \rightarrow \ty_2
}

\infer{
  \freeinenv{\tyvar} \\
  \tyenvcons\tyenv\tyvar{*} \vdash \ty
}{
  \tyenv \vdash \forall \tyvar. \ty
}

\infer{
  \forall i \in [1,n] \; \tyenv \vdash \ty_i \\
  ar(\epsilon) = |\vec{\ty}_n|
}{
  \tyenv \vdash \epsilon \vec{\ty}_n
}

\infer{
  \vdash \tyenv \\
  \binding{\var}{\{\var\}_{\dtype}} \in \tyenv
}{
  \tyenv \vdash \{\var\}_{\dtype}
}

\infer{
  \forall i \in [1,n] \; \tyenv \vdash \ty_i
}{
  \tyenv \vdash \{ n: \vec{\ty_n}\}
}
\end{mathpar}

\subsection{Règles de typage}

\begin{mathpar}
%
\infer{
  \vdash \tyenv \\
  \tyenvmem{\binding\var\ty}\tyenv
}{
  \tycheck\tyenv\var\ty
}\rname{T-Var}

\infer{
  \freeinenv{\var} \\
  \tycheck{\tyenvcons\tyenv\var{\ty_1}, \csenv}{\term}{\ty_2}
}{
  \tycheck{\tyenv, \csenv}{\lam{\binding\var{\ty_1}}{\term}}{\tyarrow{\ty_1}{\ty_2}}
}\rname{T-Abs}

\infer{
  \tycheck{\tyenv, \csenv}{\term_1}{\tyarrow{\ty_1}{\ty_2}} \\
  \tycheck{\tyenv, \csenv}{\term_2}{\ty_1}
}{
  \tycheck{\tyenv, \csenv}{\term_1 \term_2}{\ty_2}
}\rname{T-App}
\end{mathpar}
%
Règles de typage usuelles du lambda-calcul:
\begin{itemize}
  \item Une variable doit faire référence à un nom déjà lié à un type dans le
    contexte.
  \item Le corps d'une abstraction est typé sous un contexte augmenté du nom et
    du type liés. Le tout a un type flèche, dont le domaine est le type de
    l'argument et le codomaine celui du corps.
  \item Seule une abstraction peut être appliquer, et le domaine de
    l'abstraction et le type de l'argument doivent coïncider.
\end{itemize}

\begin{mathpar}
\infer{
  \freeinenv{\var} \\
  \tycheck{\tyenvcons{\tyenv}{\tyvar}{\tykind}, \csenv}{\term}{\ty}
}{
  \tycheck{\tyenv, \csenv}{\Lambda \tyvar . \term }{\forall \tyvar.\, \ty}
}\rname{T-TAbs}

\infer{
  \tycheck{\tyenv, \csenv}{\term}{\forall \tyvar.\, \ty_1}
}{
  \tycheck{\tyenv, \csenv}{\term \, \ty_2}{\ty_1 \{\tyvar \leftarrow \ty_2\}}
}\rname{T-TApp}
\end{mathpar}
%
Extension du lambda-calcul polymorphe : les règles précédentes d'abstraction et
d'application de termes sont transposées aux types.

La quantification universelle remplace les types flèches pour indiquer
l'abstraction par rapport à un type.

\begin{mathpar}
\infer{
  \freeinenv{\var} \\
  \freeinenv{y} \\
  \var \notin \kw{FV}(\ty) \\
  y \notin \kw{FV}(\ty) \\
  \tycheck{\tyenv, \csenv}{\term_1}{\dtype} \\
  \tycheck{\tyenvcons{\tyenvcons{\tyenv}
                     {\var}{ \{ \var \}_{\dtype} }}
                     {y}{ \{ y \}_{int} },
           \csenv \wedge \kw{tag} \var = y}
          {\term_2}{\ty}
}{
  \tycheck{\tyenv, \csenv}{\kw{unpack} \term_1 \kw{as} y : \var \kw{in} \term_2}{\ty}
}\rname{T-Unpack}
\end{mathpar}
%
L'extraction d'une étiquette se fait en liant deux nouvelles variables de
termes dans l'environnement. 

Ces variables de termes apparaissant également dans les types -- c'est le seule
construction introduisant des types \emph{singleton}s -- il faut s'assurer que
ces noms ne s'échappent pas de la construction $\kw{unpack}$.

Enfin, le typage se fait sur le corps du $\kw{unpack}$ sous la contrainte que
l'étiquette de $\var$ est bien décrite par $y$.

\begin{mathpar}
%
\infer{
  k \in dom({\dtype}) \\
  \tycheck{\tyenv, \csenv \wedge \kw{tag} \var \in \{ k \} \wedge \Phi(\dtype, k)}{\term}{\ty} \\
  \tycheck{\tyenv, \csenv \wedge \kw{tag} \var \in \lnot \{ k \}, \var, \dtype}{\vec{b}}{\ty}
}{
  \tycheck{\tyenv, \csenv, \var, \dtype}{k \rightarrow \term \;|\; \vec{b}}{\ty}
}\rname{T-DepPatTag}

\infer{
  \tycheck{\tyenv, \csenv}{\term}{\ty}
}{
  \tycheck{\tyenv, \csenv, x, \dtype}{\_ \rightarrow \term}{\ty}
}\rname{T-DepPatAny}

\infer{
  \csenv \Vdash \bot
}{
  \tycheck{\tyenv, \csenv, x, \dtype}{\emptyset}{\ty}
}\rname{T-DepPatEmpty}

\infer{
  \tycheck{\tyenv, \csenv}{\term}{\{ \var \}_{\dtype}} \\
  \tycheck{\tyenv, \csenv, \var, {\dtype}}{\vec{b}}{\ty}
}{
  \tycheck{\tyenv, \csenv}{\kw{switch} \term \kw{in} \vec{b}}{\ty}
}\rname{T-DepSwitch}
%
\end{mathpar}
%
Pour typer les branches, on ajoute depuis la construction $\kw{switch}$
la variable testée et le type de données auquel elle est liée dans le contexte
de typage des branches. Puis on poursuit par le typage des branches. 

\paragraph{Cas étiquetté} Le test d'une étiquette restreint la valeur
d'une étiquette par le biais des contraintes pour typer le corps de la branche
en cas de succès, ou bien exclue cette valeur de l'ensemble des étiquettes pour
typer le reste des branches.

En cas de succès, les contraintes propres à cette étiquette du type de données
sont introduites avant de poursuivre le typage du corps (elles sont extraites
\emph{via} la fonction $\Phi$).

\paragraph{Cas par défaut} Si aucune branche n'a réussit, un cas par défaut
doit être fourni. Son typage est plus simple, en particulier il n'est fait pas
usage de $\var$ ni de $\dtype$

\paragraph{Absence de cas par défaut} Si l'on peut déduire $\bot$ des
contraintes, c'est que la branche actuelle fait partie du code mort; tous les
étiquettes possibles ont été écartées avant.

\begin{mathpar}
%
\infer{
  \tycheck{\tyenv, \csenv}{\term}{\ty} \\
}{
  \tycheck{\tyenv, \csenv, k}{k \rightarrow \term \;|\; \vec{b}}{\ty}
}\rname{T-PatTag-1}

\infer{
  \tycheck{\tyenv, \csenv, k'}{\vec{b}}{\ty}
}{
  \tycheck{\tyenv, \csenv, k'}{k \rightarrow \term \;|\; \vec{b}}{\ty}
}\rname{T-PatTag-2}

\infer{
  \tycheck{\tyenv, \csenv}{\term}{\ty}
}{
  \tycheck{\tyenv, \csenv, k}{\_ \rightarrow \term}{\ty}
}\rname{T-PatAny}

\infer{
  \tyenv, \csenv \Vdash \bot
}{
  \tycheck{\tyenv, \csenv, k}{\emptyset}{\ty}
}\rname{T-PatEmpty}

\infer{
  \tycheck{\tyenv, \csenv}{\val}{\{k : \vec{ty}\}} \\
  \tycheck{\tyenv, \csenv, k}{\vec{b}}{\ty}
}{
  \tycheck{\tyenv, \csenv}{\kw{switch} \val \kw{in} \vec{b}}{\ty}
}\rname{T-Switch}
%
\end{mathpar}
%
Ces règles sont une version non dépendante des règles précédentes. Elles
servent à montrer la correction du système de type.

Lors de la réduction de $\kw{unpack}$, les variables introduites vont être
substituées. Or les règles de typage de $\kw{switch}$ font référence à ces
variables singletons. Durant cette réduction, la valeur exacte des étiquettes
est connu : les règles ci-dessus permettent de prouver que si l'on connaît
cette valeur, il est possible de réécrire la dérivation de typage sans utiliser
de types dépendants.

\subsubsection{Type des primitives}

\begin{mathpar}
\infer{
  \tycheck{\tyenv, \csenv}{\term}{\{ k: \vec{\ty_n} \}}
}{
  \tycheck{\tyenv, \csenv}{\kw{field}_i \term}{\ty_i}
}\rname{T-field-imm}

\infer{
  \tycheck{\tyenv, \csenv}{\term}{\{ \var \}_{\dtype}} \\
  \tyenv, \csenv \Vdash \kw{tag} \var \in \{ k \} \\
  fields({\dtype}, k) = \vec{\ty_n}
}{
  \tycheck{\tyenv, \csenv}{\kw{field}_i \term}{\ty_i}
}\rname{T-field}
\end{mathpar}
%
La projection d'une composante d'une valeur peut se faire soit si l'on connaît
le type immédiat du bloc, soit s'il s'agit d'un type de données dont on peut
déduire l'étiquette exacte depuis les contraintes.

\begin{mathpar}
\infer{
  k \in dom({\dtype}) \\
  n = ar({\dtype}) \\
  \tycheck{\forall i \in [1,n]. \, \tyenv, \csenv}{\term_i}{field({\dtype},k,i)} \\
}{
  \tycheck{\tyenv, \csenv}{\kw{makeblock}_{k,{\dtype}} \vec{\term_n}}{{\dtype}}
}\rname{T-makeblock}
\end{mathpar}
%
La construction d'un bloc est paramétrée par le type dont on cherche à
construire une valeur et l'étiquette de cette valeur. Il faut alors vérifier que
l'arité et les types des arguments correspondent à la définition du type.

%\infer{
%  \tycheck{\tyenv, \csenv}{\term}{\kw{Tag}_z \var} \\
%}{
%  \tycheck{\tyenv, \csenv}{\kw{shift}_k \term}{\kw{Tag}_{z+k} \var}
%}\rname{T-shift}
\begin{mathpar}
\infer{
  \tycheck{\tyenv, \csenv}{\term}{\{ \var \}_{\dtype}} \\
}{
  \tycheck{\tyenv, \csenv}{\kw{isint} \term}{\kw{IsInt}(x,{\dtype})}
}\rname{T-isint}

\infer{
  \tycheck{\tyenv, \csenv}{\term}{\{ \var \}_{\dtype}} \\
}{
  \tycheck{\tyenv, \csenv}{\kw{eq}_k \term}{\operatorname{Eq}(k,x,{\dtype})}
}\rname{T-eq}

\infer{
  \tycheck{\tyenv, \csenv}{\term}{\{ \var \}_{\dtype}} \\
}{
  \tycheck{\tyenv, \csenv}{\kw{lt}_k \term}{\operatorname{Lt}(k,x,{\dtype})}
}\rname{T-lt}

\infer{
  \tycheck{\tyenv, \csenv}{\term}{\{ \var \}_{\dtype}} \\
}{
  \tycheck{\tyenv, \csenv}{\kw{isout}_{[z1,z2]} \term}{\operatorname{IsOut}(z1,z2,\var,{\dtype})}
}\rname{T-isout}
\end{mathpar}
%
Ces quatres primitives sont typées par la construction d'un booléen
introduisant des contraintes dans l'environnement.

Leur comportement exact dépend donc de la définition des \emph{méta-constructeurs}.

\paragraph{$\kw{IsInt}(x,{\dtype})$} est généré en séparant en deux
sous-ensembles les étiquettes de $\dtype$ : $S_1$ pour les constructeurs
constants ($\forall s \in S_1 \, ar(\dtype,s) = 0$) et $S_0$ pour les
constructeurs paramétrés.  Avec les contraintes: 
\begin{itemize}
  \item $\Phi(\kw{IsInt}(x,{\dtype}),0) = \kw{tag} \var \in S_0$
  \item $\Phi(\kw{IsInt}(x,{\dtype}),1) = \kw{tag} \var \in S_1$
\end{itemize}

\paragraph{$\kw{Eq}(k,x,{\dtype})$} teste si l'étiquette de $x$ est $k$.
Les contraintes générées sont les suivantes: 
\begin{itemize}
  \item $\Phi(\kw{Eq}(k,x,{\dtype}),0) = \kw{tag} \var \notin \{k\}$
  \item $\Phi(\kw{Eq}(k,x,{\dtype}),1) = \kw{tag} \var \in \{k\}$
\end{itemize}

\paragraph{$\kw{Lt}(k,x,{\dtype})$} teste si l'étiquette de $x$ est inférieur à
$k$.
Les contraintes générées sont les suivantes: 
\begin{itemize}
  \item $\Phi(\kw{Lt}(k,x,{\dtype}),0) = \kw{tag} \var \notin [0,k]$
  \item $\Phi(\kw{Lt}(k,x,{\dtype}),1) = \kw{tag} \var \in [0,k]$
\end{itemize}

\paragraph{$\kw{IsOut}(z_1,z_2,x,{\dtype})$} teste si l'étiquette de $x$ est
dans l'intervalle $[z_1,z_2]$.
fes contraintes générées sont les suivantes: 
\begin{itemize}
  \item $\Phi(\kw{IsOut}(z_1,z_2,x,{\dtype}),0) = \kw{tag} \var \in [z_1,z_2]$
  \item $\Phi(\kw{IsOut}(z_1,z_2,x,{\dtype}),1) = \kw{tag} \var \notin [z_1,z_2]$
\end{itemize}

\paragraph{Remarques} Les constructions \emph{Eq}, \emph{Lt} et \emph{IsOut}
sont redondantes. Tous les cas peuvent se ramener à \emph{IsOut} en choisissant
correctement l'intervalle. Cependant les trois primitives sont utilisées par la
VM OCaml; les prendre en charge directement permet de rester proche du langage
intermédiaire.

\subsection{Interprétation de la substitution par une valeur}
\newcommand\subst{\{\var \leftarrow v\}}

L'opération de réécriture essentielle à la preuve du lemme de substitutivité
est la substitution d'une variable de terme -- dont le type est un singleton --
par une valeur.

Voici la définition de l'opération pour les différents objets manipulés dans le
lemme.

$\var$ est substitué par une valeur $v$.
Dans le contexte depuis lequel est extrait la variable $\var$, les hypothèses
suivantes sont vérifiées :
%
\begin{align*}
  &\tycheck{\tyenv, \csenv}{\var}{\{\var'\}_{\dtype}} \\
  &\tycheck{\tyenv, \csenv}{v}{\{k : \vec{\ty}\}}
\end{align*}
%
\subsubsection{Dans les contextes}
On peut supposer que $\var \# \tyenv$ : l'opération n'est effectuée que sur
des contextes dont on a extrait auparavant la liaison de $\var$.
\begin{align*}
                            . \subst \;&\Rightarrow\; . \\
     \tyenvcons\tyenv{y}{\ty} \subst \;&\Rightarrow\; \tyenvcons\tyenv{y}{\ty \subst} \\
  \tyenvcons\tyenv{\alpha}{*} \subst \;&\Rightarrow\; \tyenvcons\tyenv{\alpha}{*} 
\end{align*}

\subsubsection{Dans les contraintes}
\begin{align*}
         C_1 \wedge C_2 \subst \;&\Rightarrow\; C_1 \subst \wedge C_2 \subst  \\
    \term_1 = \term_2 \subst \;&\Rightarrow\; \term_1\subst = \term_2\subst \\
    \kw{tag} y    \in S \subst \;&\Rightarrow\; \kw{tag} y \in S & \text{où } \var \neq y \\
    \kw{tag} y    =  y' \subst \;&\Rightarrow\; \kw{tag} y \in S & \text{où } \var \neq y, \var \neq y' \\
    \kw{tag} x    =  y  \subst \;&\Rightarrow\; \top \\
    \kw{tag} y    =  x  \subst \;&\Rightarrow\; \kw{tag} x \in \{k\} & \text{où } v : \{k : \vec{\ty} \} \\
                 \top \subst \;&\Rightarrow\; \top \\
                 \bot \subst \;&\Rightarrow\; \bot
\end{align*}

Pour la substitution de $\kw{tag} \var \in S \subst$, la réécriture se fait
suivant l'étiquette de $v \,:\, \{k : \vec{\ty}\}$. \\
Si $k \in S$, alors $\kw{tag} \var \in S \subst \;\Rightarrow\; \top$. \\
Sinon, alors $k \notin S$ et $\kw{tag} \var \in S \subst \;\Rightarrow\; \bot$.

\subsubsection{Dans les termes}
\begin{align*}
  \var
   \, \subst \;&\Rightarrow\; v
    \\
  y
   \, \subst \;&\Rightarrow\; y
    \\
  \term_1 \, \term_2
   \, \subst \;&\Rightarrow\; \term_1 \subst \, \term_2 \subst
    \\
  (\lambda \binding{y}{\ty} . \term)
   \, \subst \;&\Rightarrow\; \lambda \binding{y}{\ty \subst} . \term \subst
    \\
  (\Lambda \tyvar . \term)
   \, \subst \;&\Rightarrow\; \Lambda \tyvar . \term \subst
    \\
  \term \, \ty
   \, \subst \;&\Rightarrow\; \term \subst \, \ty \subst
    \\
  prim \, \vec{\term}
   \, \subst \;&\Rightarrow\; prim \, \vec{\term'} \; \text{où} \; \term'_i = \term_i \subst
    \\
  \kw{switch} \term \kw{in} \vec{b} : \ty
   \, \subst \;&\Rightarrow\; 
  \kw{switch} \term \subst \kw{in} \vec{b} \subst : \ty \subst 
    \\
  \kw{unpack} \term_1 \kw{as} y : x' \kw{in} \term_2
   \, \subst \;&\Rightarrow\; 
  \kw{unpack} \term_1 \subst \kw{as} y : x' \kw{in} \term_2 \subst
    \\
\end{align*}
%
Par construction, les variables liées sont fraîches et les valeurs substituées
sont closes.

On peut donc supposer que l'opération de substitution ne s'applique jamais sur
une variable liée l'intérieur d'un terme.

\subsubsection{Dans les types}
\begin{align*}
  \ty_1 \rightarrow \ty_2
   \, \subst \;&\Rightarrow\; 
    \ty_1 \subst \rightarrow \ty_2 \subst
  \\
  \forall \tyvar. \ty
   \, \subst \;&\Rightarrow\; 
  \forall \tyvar. (\ty \subst)
  \\
  \{ \var \}_{\dtype}
   \, \subst \;&\Rightarrow\; 
   {\dtype} \, v
   &\text{où}\; \var = \{ x \}_{\dtype}
  \\
  \{ n: \vec{\ty} \}
    \, \subst \;&\Rightarrow\; 
    \{ n: \vec{\ty'} \}
    &\text{où}\; \ty'_i = \ty_i \subst
  \\
  \epsilon \, \vec{\ty}
    \, \subst \;&\Rightarrow\; \epsilon \, \vec{\ty'}
    &\text{où}\; \ty'_i = \ty_i \subst
\end{align*}

\subsection{Propriétés du système de type}

La preuve du système de type s'appuie sur la technique de preuve classique
proposée par \cite{Wright92asyntactic}, qui consiste à montrer les deux
théorèmes « subject reduction » et « progress », lesquels entraînent la sûreté
du système de type.

\begin{thm}[Subject-reduction]
  Si $\tycheck{\tyenv, \csenv}{\term}{\ty}$ et $\term \Rightarrow \term'$ alors 
     $\tycheck{\tyenv, \csenv}{\term'}{\ty}$.
\end{thm}

\begin{thm}[Progress]
  Si $\tycheck{\tyenv, \csenv}{\term}{\ty}$ alors soit $\term$ est une valeur,
    soit il existe $\term'$ tel que $\term \Rightarrow \term'$.
\end{thm}

Les preuves de \emph{Subject-Reduction} et \emph{Progress} suivent une
structure établie reposant sur le lemme de la substitutivité (Préservation des
types par substitution). Nous essayons de le prouver ci-après pour les
extensions proposées par notre lambda-calcul.

\begin{lemma}[Inversion de la relation de typage]
\begin{mathpar}

\infer{
  \tycheck\tyenv\var\ty
}{
  \tyenvmem{\binding\var\ty}\tyenv
}

\infer{
  \tycheck{\tyenv, \csenv}{\lam{\binding\var{\ty_1}}{\term}}{\ty}
}{
  \ty = \tyarrow{\ty_1}{\ty_2} \\
  \tycheck{\tyenvcons\tyenv\var{\ty_1}, \csenv}{\term}{\ty_2}
}

\infer{
  \tycheck{\tyenv, \csenv}{\term_1 \term_2}{\ty_2}
}{
  \tycheck{\tyenv, \csenv}{\term_1}{\tyarrow{\ty_1}{\ty_2}} \\
  \tycheck{\tyenv, \csenv}{\term_2}{\ty_1}
}

\infer{
  \tycheck{\tyenv, \csenv}{\Lambda \tyvar . \term }{\ty}
}{
  \ty = {\forall \tyvar.\, \ty'} \\
  \tycheck{\tyenvcons{\tyenv}{\tyvar}{\tykind}, \csenv}{\term}{\ty}
}

\infer{
  \tycheck{\tyenv, \csenv}{\term \, \ty_2}{\ty}
}{
  \ty =\ty_1 \{\tyvar \leftarrow \ty_2\} \\
  \tycheck{\tyenv, \csenv}{\term}{\forall \tyvar.\, \ty_1}
}

\infer{
  \tycheck{\tyenv, \csenv}{\kw{unpack} \term_1 \kw{as} y : \var \kw{in} \term_2}{\ty}
}{
  \freeinenv{\var} \\ \freeinenv{y} \\
  x \notin \kw{FV}(\ty) \\ y \notin \kw{FV}(\ty) \\
  \tycheck{\tyenv, \csenv}{\term_1}{{\dtype}} \\
  \tycheck{\tyenvcons{\tyenvcons\tyenv{\var}{ \{ \var \}_{\dtype} }}
                     {y}{\{y\}_{int}}, \csenv \wedge \kw{tag} \var = y }
          {\term_2}{\ty}
}

\infer{
  \tycheck{\tyenv, \csenv, x, {\dtype}}{k \rightarrow \term \;|\; \vec{b}}{\ty}
}{
  k \in dom({\dtype}) \\
  \tycheck{\tyenv, \csenv \wedge \kw{tag} \var \in \{ k \} \wedge {\dtype} k}{\term}{\ty} \\
  \tycheck{\tyenv, \csenv \wedge \kw{tag} \var \in \lnot \{ k \}, x, {\dtype}}{\vec{b}}{\ty}
}

\infer{
  \tycheck{\tyenv, \csenv, x, {\dtype}}{\_ \rightarrow \term}{\ty}
}{
  \tycheck{\tyenv, \csenv}{\term}{\ty}
}

\infer{
  \tycheck{\tyenv, \csenv, x, {\dtype}}{\emptyset}{\ty}
}{
  \tyenv, \csenv \Vdash \bot
}

\infer{
  \tycheck{\tyenv, \csenv}{\kw{switch} \term \kw{in} \vec{b}}{\ty}
}{
  \tycheck{\tyenv, \csenv}{\term}{\{ \var \}_{\dtype}} \\
  \tycheck{\tyenv, \csenv, \var, {\dtype}}{\vec{b}}{\ty}
}

\end{mathpar}

\begin{proof}
  Immédiate par inversion des règles de typage,
  le système est dirigé par la syntaxe.
\end{proof}
\end{lemma}

\begin{lemma}[Préservation du type des branches par substitution d'un singleton]
\begin{mathpar}
\infer{
  \tycheck{\tyenv, \csenv, \var, \dtype}{\vec{b}}{\ty} \\
  \tycheck{\tyenv, \csenv}{\var}{\{k: \vec{v}'\}}
}{
  \tycheck{\tyenv \{\var \leftarrow \val\}, \csenv \{\var \leftarrow \val\},
    k}{\vec{b} \{\var \leftarrow \val\} }{\ty}
}
\end{mathpar}

\begin{proof}
  Par induction sur la dérivation de $\vec{b} : \ty$.
  Les trois formes possibles de jugement sont \textsc{T-Dep-PatTag},
  \textsc{T-Dep-PatAny} et \textsc{T-Dep-PatEmpty}.

\rcase{T-Dep-PatEmpty}
  Cas de base. Par inversion, on sait que $\tyenv, \csenv \Vdash \bot$.
  C'est suffisant pour pouvoir appliquer le jugement \textsc{T-PatEmpty} qui permet de
  conclure immédiatement.

\rcase{T-Dep-PatAny}
  Cas de base. Par inversion, $\vec{b} = _ \rightarrow \term$ et
  $\tycheck{\tyenv, \csenv}{\term}{\ty}$.
  On peut donc appliquer le jugement \textsc{T-PatAny} qui permet de
  conclure immédiatement.

\rcase{T-Dep-PatTag}
  Cas inductif. Hypothèses:
\begin{equation*}
\begin{aligned}
  & \tycheck{\tyenv, \csenv, \var, \dtype}{k' \rightarrow \vec{b}'}{\ty} \\
  & k' \in dom({\dtype}) \\
  & \tycheck{\tyenv, \csenv \wedge \kw{tag} \var \in \{ k' \} \wedge \Phi(\dtype, k')}{\term}{\ty} \\
  & \tycheck{\tyenv, \csenv \wedge \kw{tag} \var \in \lnot \{ k' \}, \var, \dtype}{\vec{b}'}{\ty}
\end{aligned}
\end{equation*}

  Si $k = k'$, on applique \textsc{T-PatTag-1} sur l'environnement réécrit après
  substitution de $\var$ par $\val$.

  Si $k \neq k'$, on applique \textsc{T-PatTag-2} en utilisant l'hypothèse 
  d'induction.

  \paragraph{Remarques} Il faut montrer que la substitution dans
  l'environnement satisfait les contraintes pour pouvoir appliquer les
  jugements non dépendants \ldots
\end{proof}
\end{lemma}

\begin{lemma}[Préservation des types par substitution]
\begin{mathpar}
\infer{
  \tycheck{\tyenvcons\tyenv\var{\ty'}, \csenv}{\term}{\ty} \\
  \tycheck{\tyenv, \csenv}{\val}{\ty'}
}{
  \tycheck{\tyenv \{ \var \leftarrow \val \}, \csenv \{ \var \leftarrow \val \} }{\term \{ \var \leftarrow \val \}}{\ty}
}
\end{mathpar}

\begin{proof}
Par induction sur la dérivation du jugement
  $\tycheck{\tyenvcons\tyenv\var{\ty'}, \csenv}{\term}{\ty}$.
  
\rcase{T-Unpack} 
\begin{equation*}
\begin{aligned}
  & \term = \kw{unpack} \term_1 \kw{as} y \kw{in} \term_2 \\
  & y \notin dom(\tyenvcons\tyenv\var{\ty'}) \\
  & y \notin FV(\ty) \\
  & \tycheck{\tyenvcons\tyenv\var{\ty'}, \csenv}{\term_1}{\ty_1} \\
  & \tycheck{\tyenvcons{\tyenvcons\tyenv\var{\ty'}}{y}{\ty_1}, \csenv}{\term_2}{\ty}
\end{aligned}
\end{equation*}

Par hypothèse, $y \neq \var$.
L'hypothèse d'induction nous permet de déduire 
  $\tycheck{\tyenvcons\tyenv{y}{\ty_1}, \csenv}{\term_1}{\ty_1}$.
Par permutation de contexte, on montre
  $\tycheck{\tyenvcons{\tyenvcons\tyenv{y}{\ty_1}}\var{\ty'}, \csenv}{\term_2}{\ty}$.
En affaiblissement le contexte de la seconde hypothèse, on obtient
  $\tycheck{\tyenvcons\tyenv{y}{\ty_1}, \csenv}{\val}{\ty'}$.
On peut alors appliquer l'hypothèse d'induction sur le terme $\ty_2$:
  $\tycheck{\tyenvcons\tyenv{y}{\ty_1}, \csenv}{\term_2\{\var \leftarrow \val\}}{\ty'}$.
\\
On applique alors appliquer $\rref{T-Unpack}$ qui nous donne le résultat attendu.
  $\tycheck{\tyenvcons\tyenv\var{\ty'}, \csenv}
           {\kw{unpack} \term_1 \kw{as} y \kw{in} \term_2}{\ty}$.

\rcase{T-DepSwitch}
\begin{equation*}
\begin{aligned}
  & \term = \kw{switch} \term_1 \kw{in} \vec{b} \\
  & \tycheck{\tyenvcons\tyenv\var{\ty'}, \csenv}{\term_1}{\{y\}_{\dtype}} \\
  & \tycheck{\tyenvcons\tyenv\var{\ty'}, \csenv, y, {\dtype}}{\vec{b}}{\ty}
\end{aligned}
\end{equation*}

Ici, deux cas se présentent selon la définition de $y$ : si $y \neq \var$ la
preuve est immédiate par application des hypothèses d'induction sur les
sous-termes. 

Supposons $y = \var$. Alors $\val : \ty'$ est de la forme $\{n : \vec{\val}'\}$.
En utilisant le lemme précédent, on montre
$$\tycheck{\tyenv \{ \var \leftarrow \val \}, \csenv \{ \var \leftarrow \val \} }{\vec{b} \{ \var \leftarrow \val \}}{\ty}$$
\textsc{T-Switch} permet alors de conclure.

\paragraph{Remarques} La méthode de preuve semble bonne, mais la preuve est
  incomplète et le cadre formel pour l'exprimer correctement insuffisant.

\end{proof}
\end{lemma}

