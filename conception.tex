\section{Formalisation du langage}

\subsection{Séparer extraction et analyse d'un \emph{tag}}

La principale subtilité réside entre la séparation du calcul d'un tag et du
branchement qui s'ensuivra.

\paragraph{Filtrage de motif} Le \emph{filtrage de motif} est la construction
qui permet l'analyse des valeurs de types sommes pour sélectionner une branche
d'exécution.

Les différents constructeurs d'un type algébrique mènent chacun à une branche
différente. Dans chacune, des noms liés aux paramètres du constructeur
testé sont introduits. Le programmeur n'indique jamais explicitement l'accès
aux champs d'une valeur. Il ne peut même pas nommer un champ en dehors de la
branche.

Il est alors simple de vérifier l'exhaustivité du traitement et de garantir
l'absence d'accès à des valeurs inexistantes lors de l'exécution. Enfin le
compilateur se voit offrir beaucoup de libertés pour générer du code efficace,
différentes méthodes sont étudiées dans \cite{LeFessant:2001:OPM:507669.507641}
puis \cite{Maranget:2008:CPM:1411304.1411311}.

\paragraph{Motifs compilés} Le compilateur OCaml choisit de compiler ces motifs
très tôt. Dans le langage \emph{Lambda} il n'existe déjà plus que deux formes
de tests, les \emph{switch} et \emph{if} que l'on retrouve dans les langages
impératifs classiques. Une forme spécialisée de branchement proche du
\emph{goto} est aussi proposée pour factoriser le code des branches.

Les filtrages de haut niveau sont ainsi traduits via des arbres de décisions,
au travers d'une suite de tests et de tables de sauts.  Le lien entre la
sélection d'une branche et les hypothèses qui y ont amené disparaît -- le
compilateur ne peut que suivre les instructions de projection sous l'hypothèse
que le code traduit est correct.

Voici le corps de la forme compilée de la fonction \emph{resolve} définie
ci-dessus :

\lstset{language=Lisp}
\begin{lstlisting}
(switch* addr
   ;; Any
   case int 0: [0: 0 0 0 0]
   ;; Localhost
   case int 1: [0: 127 0 0 1]
   ;; IP
   case tag 0:
    (makeblock 0 (field 0 addr) (field 1 addr)
      (field 2 addr) (field 3 addr))
   ;; Host
   case tag 1: ...)
\end{lstlisting}
\lstset{language=Caml}

Le lien entre une branche et la sélection qui y a mené est devenu implicite.
En particulier les projections dans la branche "IP", matérialisée par la
primitive \emph{field}, sont impossibles à vérifier.

Notre travail sur \emph{Lambda} vise à justifier la compilation de ces motifs ;
pour cela il faudra être capable de corréler statiquement la valeur d'un tag et
le type de la valeur \emph{étiquetée}.
