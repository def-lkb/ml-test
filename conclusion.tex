\section{Conclusion}

Le sujet qui a finalement été traité dans ce TRE était inattendu et s'est
imposé naturellement après avoir expérimenter plusieurs pistes autour de
l'analyse statique.

\paragraph{Résultats de la recherche} Les résultats obtenus, bien qu'incomplets
, sont très encourageants. De nombreux prototypes ont été développés en
parallèle du projet et ont permis de mettre en évidence des difficultés
calculatoires avec les algorithmes impliqués et également des insuffisances du
système de type pour couvrir correctement le code généré par le compilateur
OCaml.

Un cas n'est toujours pas couvert par le système de type proposé : le décalage
des étiquettes par une constante utilisé pour simplifier certains tests.
Une version précédente du système de types prenait en charge ce cas, mais
ce dernier a été simplifié pour rendre la formalisation plus accessible.

Malgré cela, la quantité de travail et de technique qui sépare une version
pratique d'une version formelle et prouvée ne m'a pas permis de mener à terme
cette étude.  Par manque de temps la preuve du système actuel reste donc très
incomplète, mais les résultats intermédiaires laissent à penser que la piste
suivie devrait aboutir à une solution au problème considéré.

En parallèle de ce travail manuel, un début de formalisation à l'aide de
\emph{Coq} a été entrepris, en suivant le cours \emph{"Software Foundations"}
de Pierce. L'objet de ce rapport pourrait donc être poursuivi, notamment en
apportant une preuve formelle du système de type puis une implantation
certifiée du typeur.

\paragraph{Apprentissage personnel} Dans un premier temps découvrir
l'architecture du compilateur OCaml a été une expérience stimulante en tant que
telle. Mais c'est surtout le travail avec des outils rigoureux et bien établis
de la théorie des langages de programmation -- par ordre de découverte,
\mbox{Système-F}, la méta-théorie d'un langage et enfin son implantation en
\emph{Coq} -- qui a constitué la partie la plus enrichissante de ce projet de
recherche.
