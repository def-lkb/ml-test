\section{Conclusion}

Le sujet qui a finalement été traité dans ce TRE était inattendu et s'est
imposé naturellement après plusieurs avoir expérimenter plusieurs pistes autour
de l'analyse statique.

\paragraph{Résultats de la recherche}
Les résultats obtenus, bien qu'incomplets dans l'état actuel, sont très
encourageants. De nombreux prototypes ont été développés en parallèle du projet
et ont permis de mettre en évidence des difficultés calculatoires avec les
algorithmes impliqués et également des insuffisances du système de type pour
couvrir correctement le code généré par le compilateur OCaml.

Par ailleurs, un cas n'est toujours pas couvert par le système de type proposé :
le décalage des étiquettes par une constante utilisé pour simplifier certains
branchements.
Une version précédente du système de types prenait en charge ce
cas, mais celui-ci a été simplifié pour rendre la formalisation plus
accessible.

Malgré cela, la quantité de travail et de technique qui sépare une version
pratique d'une version formelle et prouvée ne m'a pas permis de mener à terme
cette étude.  Par manque de temps la preuve du système actuel reste donc très
incomplète, mais les résultats intermédiaires laissent à penser que la piste
suivie devrait aboutir à une solution au problème évoqué.

\paragraph{Apprentissage personnel}
