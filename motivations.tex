\chapter{Motivations}

Ce TRE a débuté par une étude de la faisabilité d'une analyse statique du
flot des exceptions pour OCaml 4.00. 

Les systèmes d'exceptions permettent au programmeur de manipuler le contrôle de
flot beaucoup plus librement qu'avec un traditionnel appel de fonction
retournant une valeur. Malheureusement, il devient alors difficile de connaître
les différents flots d'exécutions possibles car ceux-ci ne correspondent plus
strictement à la structure du code source -- on parle de contrôle de flot
non-local.

Les programmes sûrs issus d'OCaml font partie des principaux arguments en sa
faveur. Mais pour garantir cette sûreté, il faut souvent se priver
d'exceptions.  L'analyse statique que nous souhaitions effectuer visait à
vérifier de manière mécanique l'absence d'exceptions ou bien l'exhaustivité de
leur traitement dans un code source.

Ce travail reprenait les résultats de la thèse de François Pessaux
\cite{ExcAnalysis} qui avait abouti à un prototype permettant ce type d'analyse
pour OCaml 3.00.  Les résultats étaient très encourageants au point de
présenter un intérêt immédiat pour des applications industrielles.

Malheureusement, ce prototype n'a pas été porté vers les versions suivantes
d'OCaml et nous sommes arrivé à la conclusion que ce travail était trop
conséquent pour s'inscrire dans le cadre d'un TRE.  De plus, si l'approche
abordée dans la thèse permet une analyse fine, l'implantation est difficile à
maintenir car celle-ci s'appuie sur la syntaxe abstraite d'OCaml et implique de
développer en parallèle un typeur spécialisé.

Notre travail de recherche s'est alors orienté vers le langage intermédiaire du
compilateur OCaml. 

\section{\emph{Lambda}, le langage intermédiaire}

Cibler \emph{Lambda} apporte de nombreux avantages, notamment :
\begin{description}
  \item[Indépendance vis-à-vis du \emph{front-end}.] Si le langage de surface
    OCaml est régulièrement étendu, \emph{Lambda} est très stable et n'a pas
    connu de changements majeurs depuis l'introduction du système objet il y a
    plus de 10 ans. C'est une cible pérenne.(\todo{ref OO})
  \item[Langage minimaliste.]
    Les nombreuses constructions de surface sont réduites à un petit ensemble.
		La définition de \emph{lambda} tient en une dizaine de lignes ;
		la définition du langage de surface en occupe des centaines.
    L'analyse peut ainsi se concentrer directement sur le cœur du langage et
    garder une taille modeste. Le travail de maintenance est moindre.
\end{description}

Mais \emph{lambda} n'est pas typé. Dans la chaîne de traitement du
compilateur, les types sont vérifiés pour le langage source par le typeur puis
effacés. Tout le reste du traitement s'effectue sur une version non-typée du
programme. Ce choix est justifié par l'omniprésence du polymorphisme
paramétrique dans ce langage et de la représentation très régulière des valeurs
à l'exécution.

Le polymorphisme paramétrique permet de garantir qu'un programme bien typé est
sémantiquement équivalent après effacement des types \todo{ref didier rémy ?}.
La représentation des valeurs permet de travailler correctement sans
information de types -- cela concerne par exemple la convention d'appel des
fonctions ou le passage du glaneur de cellule.

\todo{figure : pipeline ocaml}.

Ce choix restreint les traitements possibles dans la suite de la compilation.
Le compilateur natif tente par exemple de réinférer des types dans le cadre de
certaines optimisations, mais il ne s'agit que d'approximation conservative.
Dans notre cas analyser le flot devient beaucoup plus compliqué -- un travail
absurde quand on sait que les informations nécessaires pour guider cette
analyse viennent d'être effacées dans la chaîne de compilation.

Concevoir une version de \emph{lambda} préservant les types semble être un
prérequis pour un outil d'analyse viable.

\section{Vers une version \emph{typée} de \emph{Lambda}}

Le langage \emph{lambda} est un lambda-calcul proposant des fonctions de très
bas-niveau.

Dans le cadre de la compilation, le lambda-calcul désigne une famille de
langage de programmation fonctionnelle reposant autour d'un unique mécanisme
d'abstraction.  Cela rend sa définition simplissime, le noyau syntaxique tient
en trois règles.  Mais derrière celles-ci se cache une très grande expressivité
: c'est un langage turing-complet et d'un suffisamment haut-niveau pour
permettre un encodage léger des constructions classiques des langages de
programmation \cite{LTU}.

Un lambda-calcul est ainsi un choix naturel pour un langage intermédiaire. La
variante d'OCaml ajoute principalement des primitives d'accès mémoires et de
branchements; des extensions de bas-niveau.

Une autre implantation majeure du langage de programmation fonctionnelle
Haskell, GHC \cite{Marlow98thenew}, a fait un choix très différent dans la
conception de son langage intermédiaire nommé \emph{Core}
\cite{Sulzmann07systemf}.  Ce dernier est typé et n'offre pas de telles
primitives; les accès mémoires sont masqués derrière une forme simplifiée de
\emph{filtrage de motifs}. Ce design
influencera nos choix par la suite.

\section{Le typage des constructions de bas-niveau}

Les systèmes de types d'OCaml et d'Haskell présentent de nombreuses
similitudes. La réussite de \emph{Core} nous a conforté dans la faisabilité
d'un traitement similaire pour OCaml.

Dans le souci d'être le moins intrusif possible dans le compilateur
OCaml actuel, il était nécessaire d'étendre le langage \emph{lambda} et non de
le remplacer.

Pour mener à terme cet objectif, les étapes suivantes se profilaient :
\begin{itemize}
  \item concevoir un système de type s'inspirant de System $F_c$ et composant
    avec les contraintes imposées par \emph{lambda},
  \item établir encodage des constructions d'Ocaml (types algébriques, modules,
    GADTs, variants polymorphes) dans ce système afin de s'assurer que toutes
    sont capturées,
  \item étendre \emph{lambda} et proposer un schéma de compilation vers ce
    nouveau langage intermédiaire.
\end{itemize}

Ceci constitue un travail de grande ampleur. Dans ce rapport nous adressons le
point précis de typer les contraintes imposées par \emph{lambda}; un
branchement et des accès mémoires beaucoup plus libres que ce que ne permet le
\emph{filtrage de motif} traditionnel.

